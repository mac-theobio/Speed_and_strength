\documentclass[12pt]{article}
\usepackage[utf8]{inputenc}
\usepackage{graphicx}
\usepackage{grffile}
\usepackage{color}
\usepackage[top=1in,left=1in, right = 1in, footskip=1in]{geometry}

\usepackage{tabularx}

\usepackage{amsmath}
\usepackage{natbib}
\usepackage{hyperref}
\bibliographystyle{plain}
\date{\today}
\thispagestyle{empty}

\usepackage{bm}

\usepackage{afterpage}
\usepackage{pdflscape}

\newcommand{\etal}{\textit{et al.}}

\newcommand{\comment}{RENEW the comment command}
\renewcommand{\comment}[3]{}
\renewcommand{\comment}[3]{\textcolor{#1}{\textbf{[#2: }\textit{#3}\textbf{]}}}

\newcommand{\jd}[1]{\comment{cyan}{JD}{#1}}
\newcommand{\swp}[1]{\comment{magenta}{SWP}{#1}}

\newcommand{\Rx}[1]{\ensuremath{{\mathcal R}_{#1}}} 
\newcommand{\Ro}{\Rx{0}}
\newcommand{\RR}{\ensuremath{{\mathcal R}}}
\newcommand{\Rhat}{\ensuremath{{\hat\RR}}}
\newcommand{\Rpost}{\Rx{post}}

\newcommand{\rr}{\ensuremath{{r}}}
\newcommand{\rhat}{\ensuremath{{\hat\rr}}}
\newcommand{\rx}[1]{\ensuremath{\rr_{\mathrm{#1}}}}
\newcommand{\rpre}{\rx{pre}}
\newcommand{\rpost}{\rx{post}}

\newcommand{\tsub}[2]{#1_{{\textrm{\tiny #2}}}}
\newcommand{\pEarly}{\ensuremath{\tsub{p}{early}}}

\newcommand{\figref}[1]{Fig.~\ref{fig:#1}}
\newcommand{\figlab}[1]{\label{fig:#1}}
% \newcommand{\eqref}[1]{(\ref{eq:#1})}
\newcommand{\eqlab}[1]{\label{eq:#1}}
\begin{document}

\begin{flushleft}
	{\Large \textbf\newline{
		Strength and speed of an epidemic intervention
	}}
	\newline{Supplementary materials}
	\newline
	Jonathan Dushoff,
	Sang Woo Park
\end{flushleft}

\section{Supplementary Text}

Here, we use a compartmental model to characterize the effects of presymptomatic transmission on different interventions.
Assuming that infected individuals go through three stages of infection (exposed $E$, presymptomatic $I_p$, and symptomatic $I_s$) until recovery for durations of $D_e$, $D_p$, and $D_s$, respectively, the rate at which secondary cases are generated by an individual infected $\tau$ time units ago is given by:
\begin{equation}
k(\tau) = \beta_p I(D_e \leq \tau < D_e + D_p) + \beta_s I(D_e + D_p \leq \tau < D_e + D_p + D_s),
\end{equation}
where $\beta_p$ and $\beta_s$ represent pre-symptomatic and symptomatic transmission rates, and $I$ represents the indicator function.
Then, the population-level kernel for an average infected individual is given by integrating across distributions of $D_e$, $D_p$, and $D_s$:
\begin{equation}
K(\tau) = \beta_p P(D_e \leq \tau < D_e + D_p) + \beta_s P(D_e + D_p \leq \tau < D_e + D_p + D_s),
\end{equation}
where $P$ represents probability.
Dividing the population-level kernel by the reproduction number $\Ro = \beta_p E(D_p) + \beta_s E(D_s)$ yields the generation-interval distribution:
\begin{equation}
g(\tau) = \frac{p P(D_e \leq \tau < D_e + D_p)}{E(D_p)} + \frac{(1-p) P(D_e + D_p \leq \tau < D_e + D_p + D_s)}{E(D_s)},
\end{equation}
where $p = \beta_p E(D_p)/\Ro$ represents the proportion of pre-symptomatic transmission.
 
We consider a specific example in which each stage consists of two sub-compartments.
We assume that the mean duration of each stage of infection is given by $1/\sigma$, $1/\gamma_p$, and $1/\gamma_s$, respectively.
Then, the corresponding set of differential equations is given by:
\begin{equation}
\begin{aligned}
\frac{dS}{dt} &= - \left(\beta_p (I_{p1} + I_{p2}) + \beta_s (I_{s1} + I_{s2})\right) S \\
\frac{dE_1}{dt} &=   \left(\beta_p (I_{p1} + I_{p2}) + \beta_s (I_{s1} + I_{s2})\right) S - 2 \sigma E_1 \\
\frac{dE_2}{dt} &= 2 \sigma E_1 - 2 \sigma E_2\\
\frac{dI_{p1}}{dt} &= 2 \sigma E_2 - 2 \gamma_p I_{p1}\\
\frac{dI_{p2}}{dt} &= 2 \gamma_p I_{p1} - 2 \gamma_p I_{p2}\\
\frac{dI_{s1}}{dt} &= 2 \gamma_p I_{p2} - 2 \gamma_s I_{s1}\\
\frac{dI_{s2}}{dt} &= 2 \gamma_s I_{s1} - 2 \gamma_s I_{s2}\\
\frac{dR}{dt} &= 2 \gamma_s I_{s2}\\
\end{aligned}
\end{equation}
For simplicity, we assume $1/\sigma = 1/\gamma_p = 1/\gamma_s = 2.5 \textrm{ days}$ and numerically calculate $g(\tau)$.

We model symptom-based intervention as a constant hazard $h$ applied to symptomatic individuals. Then, the post-intervention differential equations for symptomatic individuals can be written as:
\begin{equation}
\begin{aligned}
\frac{dI_{s1}}{dt} &= 2 \gamma_p I_{p2} - 2 \gamma_s I_{s1} - h I_{s1}\\
\frac{dI_{s2}}{dt} &= 2 \gamma_s I_{s1} - 2 \gamma_s I_{s2}  - h I_{s2}\\
\end{aligned}
\end{equation}
Then, the post-intervention reproduction number is given by:
\begin{equation}
\Rpost = \frac{\beta_p}{\gamma_p} + \frac{\beta_s}{2 \gamma_s + h} + \frac{\beta_s \gamma_s}{(2 \gamma_s + h)^2}.
\end{equation}
The strength of intervention is then given by $\Ro/\Rpost$.
In order to calculate the speed of intervention, we simulate the model with and without the intervention in a fully susceptible population (assuming $S(0) = 1-10^{-8}$, $E_1(0) = 10^{-8}$, and all other state variables at 0) and measuring two growth rates $\rpre$ and $\rpost$.
The growth rates are calculated by regressing the log incidence between day 20 and 40 against time.
The speed of intervention is then given by $\rpre - \rpost$.

We model infection-based intervention in a similar way as the test-and-treat intervention for HIV such that the hazard of isolation is given by:
\begin{equation}
h(\tau) =  \tsub{h}{max} (1 - \exp(- k f(\tau))),
\end{equation}
where $f(\tau)$ is a gamma probability density function with a mean of 2 days and a shape parameter of 2, and $k = 1$.
The parameter $\tsub{h}{max}$ is chosen such that the strength of intervention at $p=0.25$ is equal to $\Ro$.
Note that this hazard applies to all infected individuals, rather than just symptomatic individuals.
The strength and speed of intervention are calculated using Euler-Lotka equations, as shown in the main text.

\pagebreak

\section{Supplementary Figure}

\begin{figure}[!ht]
\includegraphics[width=\textwidth]{code/hivGens.Rout.pdf}
\caption{
The effects of assuming slower ($\pEarly=0.1$) or faster ($\pEarly=0.4$) transmission on the intrinsic generation interval $g(\tau)$ and the initial backward generation interval $b_0(\tau)$ in our HIV scenario. Parameters as in Figure 2 from the main text.
}
\end{figure}

\end{document}
