\documentclass[12pt]{article}
\usepackage[utf8]{inputenc}
\usepackage{graphicx}
\usepackage{grffile}
\usepackage{color}
\usepackage[top=1in,left=1in, right = 1in, footskip=1in]{geometry}

\usepackage{tabularx}

\usepackage{amsmath}
\usepackage{natbib}
\usepackage{hyperref}
\bibliographystyle{plain}
\date{\today}
\thispagestyle{empty}

\usepackage{bm}

\usepackage{afterpage}
\usepackage{pdflscape}

\newcommand{\etal}{\textit{et al.}}

\newcommand{\comment}{RENEW the comment command}
\renewcommand{\comment}[3]{}
\renewcommand{\comment}[3]{\textcolor{#1}{\textbf{[#2: }\textit{#3}\textbf{]}}}

\newcommand{\jd}[1]{\comment{cyan}{JD}{#1}}
\newcommand{\swp}[1]{\comment{magenta}{SWP}{#1}}

\newcommand{\Rx}[1]{\ensuremath{{\mathcal R}_{#1}}} 
\newcommand{\Ro}{\Rx{0}}
\newcommand{\RR}{\ensuremath{{\mathcal R}}}
\newcommand{\Rhat}{\ensuremath{{\hat\RR}}}

\newcommand{\rr}{\ensuremath{{r}}}
\newcommand{\rhat}{\ensuremath{{\hat\rr}}}

\newcommand{\tsub}[2]{#1_{{\textrm{\tiny #2}}}}
\newcommand{\pEarly}{\ensuremath{\tsub{p}{early}}}

\newcommand{\figref}[1]{Fig.~\ref{fig:#1}}
\newcommand{\figlab}[1]{\label{fig:#1}}
% \newcommand{\eqref}[1]{(\ref{eq:#1})}
\newcommand{\eqlab}[1]{\label{eq:#1}}
\begin{document}

\begin{flushleft}
	{\Large \textbf\newline{
		Strength and speed of an epidemic intervention
	}}
	\newline{Supplementary materials}
	\newline
	Jonathan Dushoff,
	Sang Woo Park
\end{flushleft}

\section{Supplementary Text}

Here, we use a compartmental model to characterize the effects of presymptomatic transmission on different interventions.
Assuming that infected individuals go through three stages of infection (exposed $E$, presymptomatic $I_p$, and symptomatic $I_s$) until recovery for durations of $D_e$, $D_p$, and $D_s$, respectively, the rate at which secondary cases are generated by an individual infected $\tau$ time units ago is given by:
\begin{equation}
k(\tau) = \beta_p I(D_e \leq \tau < D_e + D_p) + \beta_s I(D_e + D_p \leq \tau < D_e + D_p + D_s),
\end{equation}
where $\beta_p$ and $\beta_s$ represent pre-symptomatic and symptomatic transmission rates, and $I$ represents the indicator function.
Then, the population-level kernel for an average infected individual is given by integrating across distributions of $D_e$, $D_p$, and $D_s$:
\begin{equation}
K(\tau) = \beta_p P(D_e \leq \tau < D_e + D_p) + \beta_s P(D_e + D_p \leq \tau < D_e + D_p + D_s),
\end{equation}
where $P$ represents probability.
Dividing the population-level kernel 
 
with each stage consisting of two sub-compartments, the corresponding set of differential equations is given by:
\begin{equation}
\begin{aligned}
\frac{dS}{dt} &= - \left(\beta_p (I_{p1} + I_{p2}) + \beta_s (I_{s1} + I_{s2})\right) S \\
\frac{dE_1}{dt} &=   \left(\beta_p (I_{p1} + I_{p2}) + \beta_s (I_{s1} + I_{s2})\right) S - 2 \sigma E_1 \\
\frac{dE_2}{dt} &= 2 \sigma E_1 - 2 \sigma E_2\\
\frac{dI_{p1}}{dt} &= 2 \sigma E_2 - 2 \sigma E_2\\
\frac{dI_{p2}}{dt} &= - \beta S I\\
\frac{dI_{s1}}{dt} &= - \beta S I\\
\frac{dI_{s2}}{dt} &= - \beta S I\\
\frac{dR}{dt} &= - \beta S I\\
\end{aligned}
\end{equation}

\pagebreak

\section{Supplementary Figure}

\begin{figure}[!ht]
\includegraphics[width=\textwidth]{code/hivGens.Rout.pdf}
\caption{
The effects of assuming slower ($\pEarly=0.1$) or faster ($\pEarly=0.4$) transmission on the intrinsic generation interval $g(\tau)$ and the initial backward generation interval $b_0(\tau)$ in our HIV scenario. Parameters as in Figure 2 from the main text.
}
\end{figure}

\end{document}
